% Set the font size (11pt, for now) and paper size (e.g. letterpaper, a4paper)
\documentclass[11pt, a4paper]{article} 

%----------------------------------------------------------------------
%	PACKAGES AND OTHER DOCUMENT CONFIGURATIONS
%----------------------------------------------------------------------

\usepackage{graphicx} % Required for including pictures
% ----------
% GRAPHICS
% ----------
\makeatletter
\def\maxwidth{\ifdim\Gin@nat@width>\linewidth\linewidth\else\Gin@nat@width\fi}
\def\maxheight{\ifdim\Gin@nat@height>\textheight\textheight\else\Gin@nat@height\fi}
\makeatother
% Scale images if necessary, so that they will not overflow the page
% margins by default, and it is still possible to overwrite the defaults
% using explicit options in \includegraphics[width, height, ...]{}
\setkeys{Gin}{width=\maxwidth,height=\maxheight,keepaspectratio}

\usepackage{fancyhdr} % Allows the use of fancy headers and footers
\usepackage{tikz}

\usepackage{microtype} % For typography improvement
% For landscape PDF pages
\usepackage{pdflscape}

\usepackage{url} % Allows the use of hyperlinks
\urlstyle{same} % Allows hyperlink fonts to be the same as the mainfont
\usepackage[hidelinks]{hyperref} % Allows the use of hyperlinks and 
% removes the blue boxes around the link
$if(colorlinks)$
\PassOptionsToPackage{dvipsnames,svgnames*,x11names*}{xcolor}
$endif$

\pagestyle{empty} % Removes headers and footers
\usepackage{longtable}
\usepackage{setspace} % Allows the use of double spacing in the letter body
\usepackage{natbib}
%\setlength\parindent{1cm} % Paragraph indentation
%--------------------------------------------------------------------------
%	FONTS
%--------------------------------------------------------------------------
% -------
% FONTS
% -------
% mathtools has \underbracket and \overbracket with options for controlling 
% height and thickness, but unicode-math provides its own option-free versions 
% of these bracket macros. At one point it used to patch these macros and not 
% override mathtools' commands, but it currently does not do that anymore 
% (see https://github.com/wspr/unicode-math/issues/544). 
%
% We use unicode-math here so that we can use custom math fonts with 
% \setmathfont{}, which then messes up bracket options. But there's a solution! 
% We save mathtools' brackets  after loading mathtools, then load 
% unicode-math which clobbers them, and then *after* \begin{document}, redifine 
% the stored brackets (this part is key! According to 
% https://tug.org/pipermail/lualatex-dev/2011-August/001295.html, the unicode 
% table from unicode-math is executed after beginning the document, so 
% redefining the brackets in the preamble won't work)

% Math stuff
\usepackage{amsmath, amssymb, amsfonts, amsthm, mathtools}

% Save mathtools' brackets and braces
\let\normalunderbracket=\underbracket
\let\normaloverbracket=\overbracket

\usepackage{unicode-math}  % For custom math fonts

% Custom fonts
\usepackage{fontspec}
\usepackage{xunicode}
\defaultfontfeatures{Mapping=tex-text,Ligatures=TeX,Scale=MatchLowercase}
$if(mainfont)$
\setmainfont[Numbers={Proportional,OldStyle}]{$mainfont$}
$else$
\setmainfont[Numbers={Proportional,OldStyle}]{Linux Libertine O}
$endif$
$if(sansfont)$
\setsansfont{$sansfont$}
$else$
\setsansfont{Source Sans Pro}
$endif$
$if(monofont)$
\setmonofont[Mapping=tex-ansi, Scale=MatchLowercase]{$monofont$}
$else$
\setmonofont[Mapping=tex-ansi, Scale=MatchLowercase]{Inconsolata}
$endif$
$if(mathfont)$
\setmathfont{$mathfont$}
$else$
\setmathfont{Libertinus Math}
$endif$
%----------------------------------------------------------------------------
%	DOCUMENT MARGINS
%----------------------------------------------------------------------------

% \usepackage{geometry} % Required for adjusting page dimensions
% 
% \geometry{
%     headheight = 0.1in, % Header height
% 	top=.5in, % Top margin
% 	bottom=1.5cm, % Bottom margin
% 	left=3cm, % Left margin
% 	right=3cm, % Right margin
% 	% showframe, % Uncomment to show how the type block is set on the page
% }



%-----------------------------------------------------------------------------
%	AUTHOR AND RECIPIENTS NEW COMMANDS AND DETAILS STRUCTURE
%-----------------------------------------------------------------------------

\newcommand{\authordetails}[1]{\renewcommand{\authordetails}{#1}}
\newcommand{\recipientdetails}[1]{\renewcommand{\recipientdetails}{#1}}

%-----------------------------------------------------------------------------
%	HEADER STRUCTURE
%-----------------------------------------------------------------------------
\usepackage{fontawesome5}



%
% blockquote
%
\definecolor{blockquote-border}{RGB}{221,221,221}
\definecolor{blockquote-text}{RGB}{119,119,119}
\usepackage{mdframed}
\newmdenv[rightline=false,bottomline=false,topline=false,linewidth=3pt,linecolor=blockquote-border,skipabove=\parskip]{customblockquote}
\renewenvironment{quote}{\begin{customblockquote}\list{}{\rightmargin=0em\leftmargin=0em}%
\item\relax\color{blockquote-text}\ignorespaces}{\unskip\unskip\endlist\end{customblockquote}}


% Don't typeset URLs in a monospaced font
% This has to come after apacite because apacite sets its own URL style
\urlstyle{same}



%------------------------------------------------------------------------------
\pagestyle{plain}

\begin{document}
\setstretch{1.25}

\title{$title$ \\
\textbf{$subtitle$} \\
$if(manuscript-number)$
{\footnotesize {[}Journal Manuscript No. $manuscript-number${]}}\\
$endif$}
\date{}
\author{}

\maketitle


$body$


%%% BibTeX style.
\bibliographystyle{econ}

%% BibTeX database file.
\addcontentsline{toc}{section}{References}
\bibliography{$bibliography$}

\end{document}
